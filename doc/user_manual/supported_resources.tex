%%%%%%%%%%%%%%%%%%%%%%%%%%%%%%%%%%%%%%%%%%%%%%%%%%%%%%%%%%%%%%%%%%%%%%%%%%%%%%%
%
%    NEPI, a framework to manage network experiments
%    Copyright (C) 2013 INRIA
%
%    This program is free software: you can redistribute it and/or modify
%    it under the terms of the GNU General Public License as published by
%    the Free Software Foundation, either version 3 of the License, or
%    (at your option) any later version.
%
%    This program is distributed in the hope that it will be useful,
%    but WITHOUT ANY WARRANTY; without even the implied warranty of
%    MERCHANTABILITY or FITNESS FOR A PARTICULAR PURPOSE.  See the
%    GNU General Public License for more details.
%
%    You should have received a copy of the GNU General Public License
%    along with this program.  If not, see <http://www.gnu.org/licenses/>.
%
% Author: Alina Quereilhac <alina.quereilhac@inria.fr>
%
%%%%%%%%%%%%%%%%%%%%%%%%%%%%%%%%%%%%%%%%%%%%%%%%%%%%%%%%%%%%%%%%%%%%%%%%%%%%%%%

\section{Linux resources}

\begin{itemize}
  \item Linux Node (Clean home, etc)
  \item SSH
  \item The directory structure
  \item Traces and collection
  \item Linux Application
  \item LinuxPing, LinuxTraceroute, etc
  \item CCNx
\end{itemize}

\section{Planetlab resources}

\begin{itemize}
 \item how to get an account
 \item The vsys system
 \item python-vsys
 \item TAP/TUN/TUNNEL
 \item Note on PL inestability
 \item differences between PLE and PLC
\end{itemize}

\section{OMF resources}

\begin{itemize}
  \item available OMF testbeds
  \item how to get an account
  \item the concept of resource reservation
\end{itemize}
