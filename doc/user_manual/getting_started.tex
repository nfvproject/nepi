%%%%%%%%%%%%%%%%%%%%%%%%%%%%%%%%%%%%%%%%%%%%%%%%%%%%%%%%%%%%%%%%%%%%%%%%%%%%%%%
%
%    NEPI, a framework to manage network experiments
%    Copyright (C) 2013 INRIA
%
%    This program is free software: you can redistribute it and/or modify
%    it under the terms of the GNU General Public License as published by
%    the Free Software Foundation, either version 3 of the License, or
%    (at your option) any later version.
%
%    This program is distributed in the hope that it will be useful,
%    but WITHOUT ANY WARRANTY; without even the implied warranty of
%    MERCHANTABILITY or FITNESS FOR A PARTICULAR PURPOSE.  See the
%    GNU General Public License for more details.
%
%    You should have received a copy of the GNU General Public License
%    along with this program.  If not, see <http://www.gnu.org/licenses/>.
%
% Author: Alina Quereilhac <alina.quereilhac@inria.fr>
%
%%%%%%%%%%%%%%%%%%%%%%%%%%%%%%%%%%%%%%%%%%%%%%%%%%%%%%%%%%%%%%%%%%%%%%%%%%%%%%%

NEPI is written in Python, so you will need to install Python before 
being able to run experiments with NEPI. 
NEPI is known to work on Linux (Fedore, Debian, Ubuntu) and Mac (OS X).

\section{Dependencies}

Dependencies for NEPI vary according to the features you want to enable.
Make sure the following dependencies are correctly installed in your system
before using NEPI. \\

Mandatory dependencies:
\begin{itemize}
    \item Python 2.6+
    \item Mercurial 
\end{itemize}

Optional dependencies:
\begin{itemize}
    \item SleekXMPP - Required to run experiments on OMF testbeds
\end{itemize}

\subsection{Install dependencies on Debian/Ubuntu}

\begingroup
    \fontsize{10pt}{12pt}\selectfont

\begin{verbatim}
    $ sudo aptitude install -y python mercurial
\end{verbatim}

\endgroup

\subsection{Install dependencies on Fedora}

\begingroup
    \fontsize{10pt}{12pt}\selectfont

\begin{verbatim}
    $ sudo yum install -y python mercurial
\end{verbatim}

\endgroup

\subsection{Install dependencies on Mac}

First install homebrew (\url{http://mxcl.github.io/homebrew/}),
then install Python.

\begingroup
    \fontsize{10pt}{12pt}\selectfont

\begin{verbatim}
    $ brew install python
\end{verbatim}

\endgroup

\subsection{Install SleekXMPP}

You will need \textit{git} to get the SleekXMPP sources.

\begingroup
    \fontsize{10pt}{12pt}\selectfont

\begin{verbatim}
    $ git clone -b develop git://github.com/fritzy/SleekXMPP.git
    $ cd SleekXMPP
    $ sudo python setup.py install
\end{verbatim}

\endgroup

\section{The source code}

To get NEPI's source code you will need Mercurial version 
control system. The Mercurial NEPI repo can also be browsed online at: \\

\url{http://nepi.inria.fr/code/nepi/} 

\subsection{Clone the repo}

\begingroup
    \fontsize{10pt}{12pt}\selectfont

\begin{verbatim}
$ hg clone http://nepi.inria.fr/code/nepi -r nepi-3.0-release
\end{verbatim}

\endgroup

\section{Install NEPI in your system}

You don't need to install NEPI in your system to be able to run 
experiments. However this might be convenient if you don't 
plan to modify or extend the sources.

To install NEPI, just run \emph{make install} in the NEPI source
folder.

\begingroup
    \fontsize{10pt}{12pt}\selectfont

\begin{verbatim}
    $ cd nepi
    $ make install 
\end{verbatim}

\endgroup

If you are developing your own NEPI extensions, the installed 
NEPI version might interfere with your work.
In this case it is probably more convenient to tell
Python where to find the NEPI sources, using the PYTHONPATH
environmental variable, when you run a NEPI script.

\begingroup
    \fontsize{10pt}{12pt}\selectfont

\begin{verbatim}
    $ PYTHONPATH=$PYTHONPATH:<path-to-nepi>/src python experiment.py
\end{verbatim}

\endgroup

\section{Run experiments}

There are two ways you can use NEPI to run your experiments.
The first one is writing a Python script, which will import
NEPI libraries, and run it. 
The second one is in interactive mode by using Python console.

\subsection{Run from script}

Writing a simple NEPI expeiment script is easy.
Take a look at the example in the FAQ section \ref{faq:ping_example}.
Once you have written down the script, you can run it using
Python. Note that since NEPI is not yet installed in your system,
you will need to export the path to NEPI's source code to 
the PYTHONPATH environment variable, so that Python can find
NEPI's libraries.

\begingroup
    \fontsize{10pt}{12pt}\selectfont

\begin{verbatim}
    $ export PYTHONPATH=<path-to-nepi>/src:$PYTHONPATH
    $ python first-experiment.py 
\end{verbatim}

\endgroup

\subsection{Run NEPI interactively}

The IPython console can be used as an interactive interpreter to 
execute Python instructions. We can take advantage of this feature, 
to interactively run NEPI experiments. 
We will use the IPython console for the example below. 

You can easily install IPython on Debian, Ubuntu, Fedora or Mac as follows:\\

\textbf{Debian/Ubuntu}

\begingroup
    \fontsize{10pt}{12pt}\selectfont

\begin{verbatim}

$ sudo apt-get install ipython

\end{verbatim}

\endgroup

\textbf{Fedora}\\

\begingroup
    \fontsize{10pt}{12pt}\selectfont

\begin{verbatim}

$ sudo yum install ipython

\end{verbatim}

\endgroup

\textbf{Mac}\\

\begingroup
    \fontsize{10pt}{12pt}\selectfont

\begin{verbatim}

$ pip install ipython

\end{verbatim}

\endgroup

Before starting, make sure to add Python and IPython source directory 
path to the PYTHONPATH environment variable

\begingroup
    \fontsize{10pt}{12pt}\selectfont

\begin{verbatim}

$ export PYTHONPATH=$PYTHONPATH:/usr/local/lib/python:/usr/local/share/python/ipython

\end{verbatim}

\endgroup

Then you can start IPython as follows: 

\begin{verbatim}
$ export PYTHONPATH=<path-to-nepi>/src:$PYTHONPATH
$ ipython
Python 2.7.3 (default, Jan  2 2013, 13:56:14) 
Type "copyright", "credits" or "license" for more information.

IPython 0.13.1 -- An enhanced Interactive Python.
?         -> Introduction and overview of IPython's features.
%quickref -> Quick reference.
help      -> Python's own help system.
object?   -> Details about 'object', use 'object??' for extra details.

\end{verbatim}

If you want to paste many lines at once in IPython, you will need 
to type \emph{\%cpaste} and finish the paste block with \emph{\-\-}.

The first thing we need to do to describe an experiment with NEPI 
is to import the NEPI Python modules. 
In particular we need to import the ExperimentController class. 
To do this type the following in the Python console: 

\begin{lstlisting}[language=Python]

from nepi.execution.ec import ExperimentController

\end{lstlisting}

After importing the ExperimentController class, it is possible to 
create a new instance of an the ExperimentController (EC) for 
your experiment. 
The <exp-id> argument is the name you want to give the experiment 
to identify it and distinguish it from other experiments. 

\begin{lstlisting}[language=Python]

ec = ExperimentController(exp_id = "<your-exp-id>")

\end{lstlisting}

Next we will define two Python functions: \emph{add\_node} and \emph{add\_app}.
The first one to register \textit{LinuxNodes} resources and the second one to 
register LinuxApplications resources. 

\begin{lstlisting}[language=Python]

%cpaste
def add_node(ec, hostname, username, ssh_key):
    node = ec.register_resource("LinuxNode")
    ec.set(node, "hostname", hostname)
    ec.set(node, "username", username)
    ec.set(node, "identity", ssh_key)
    ec.set(node, "cleanHome", True)
    ec.set(node, "cleanProcesses", True)
    return node

def add_app(ec, command, node):
    app = ec.register_resource("LinuxApplication")
    ec.set(app, "command", command)
    ec.register_connection(app, node)
    return app
--

\end{lstlisting}

The method \textit{register\_resource} registers a resource instance with the 
ExperimentController. The method \textit{register\_connection} indicates
that two resources will interact during the experiment. 
Note that invoking \textit{add\_node} or \textit{add\_app} has no effect other
than informing the EC about the resources that will be used during the experiment.
The actual deployment of the experiment requires the method \textit{deploy} to
be invoked.

The \textit{LinuxNode} resource exposes the hostname, username and identity 
attributes. This attributes provide information about the SSH credentials 
needed to log in to the Linux host. 
The \textit{hostname} is the one that identifies the physical host you want
to access during the experiment. The \textit{username} must correspond to a
valid account on that host, and the \textit{identity} attribute is the 
'absolute' path to the SSH private key in your local computer that allows you 
to log in to the host.

The \textit{command} attribute of the \textit{LinuxApplication} resource 
expects a BASH command line string to be executed in the remote host.
Apart from the \emph{command} attribute, the \emph{LinuxApplication} 
resource exposes several other attributes that allow to upload, 
compile and install arbitrary sources. 
The add\_app function registers a connection between a \textit{LinuxNode} and a 
\textit{LinuxApplication}. 

Lets now use these functions to describe a simple experiment. 
Choose a host where you have an account, and can access using SSH
key authentication. 

\begin{lstlisting}[language=Python]

hostname = "<the-hostname>"
username = "<my-username>"
identity = "</home/myuser/.ssh/id_rsa>"

node = add_node(ec, hostname, username, ssh_key)
app = add_app(ec, "ping -c3 nepi.inria.fr",  node)

\end{lstlisting}

The values returned by the functions add\_node and add\_app are global 
unique identifiers (guid) of the resources that were registered with the EC. 
The guid is used to reference the ResourceManager associated to a registered
resource (for instance to retrieve results or change attribute values).

Now that we have registered some resources, we can ask the ExperimentController
(EC) to deploy them. 
Invoking the \emph{deploy} command will not only configure the 
resource but also automatically launch the applications.

\begin{lstlisting}[language=Python]

 ec.deploy()

\end{lstlisting}

After some seconds, we should see some output messages informing us about the
progress in the host deployment.
If you now open another terminal and you connect to the host using 
SSH (as indicated below), you should see that a directory for your experiment 
has been created in the host. In the remote host you will see that two NEPI 
directories were created in the \$HOME directory: \emph{nepi-src} and \emph{nepi-exp}. 
The first one is where NEPI will store files that might be re used by many 
experiments (e.g. source code, input files) . The second directory \emph{nepi-exp}, 
is where experiment specific files (e.g. results, deployment scripts) will be stored. 

\begingroup
    \fontsize{10pt}{12pt}\selectfont

\begin{verbatim}

$ ssh -i identity username@hostname

\end{verbatim}

\endgroup

Inside the \emph{nepi-exp} directory, you will find another directory with 
the <exp-id> assigned to your EC, and inside that directory you should find 
one directory named node-1 which will contain the files (e.g. result traces) 
associated to the LinuxNode reosurce you just deployed. 
In fact for every resource deployed associated to that host (e.g. each 
LinuxApplication), NEPI will create a directory to place files related to it. 
The name of the directory identifies the type of resources (e.g. 'node', 
'app', etc) and it is followed by the global unique identifier (guid).

We can see if a resource finished deploying by querying its state through the EC 

\begin{lstlisting}[language=Python]

ec.state(app, hr=True)

\end{lstlisting}

Once a \textit{LinuxApplication} has reached the state 'STARTED', 
we can retrieve the 'stdout' trace, which should contain the output 
of the PING command. 

\begin{lstlisting}[language=Python]

 ec.trace(app, "stdout")

\end{lstlisting}

That is it. We can terminate the experiment by invoking the method \emph{shutdown}.

\begin{lstlisting}[language=Python]

 ec.shutdown()

\end{lstlisting}

\subsection{Define a workflow}

Now that we have introduced to the basics of NEPI, we will register 
two more applications and define a workflow where one application 
will start after the other one has finished executing. 
For this we will use the EC \textit{register\_condition} method described below:

\begin{lstlisting}[language=Python]

register_condition(self, guids1, action, guids2, state, time=None):
    Registers an action START, STOP or DEPLOY for all RM on list
    guids1 to occur at time 'time' after all elements in list guids2 
    have reached state 'state'.
    
        :param guids1: List of guids of RMs subjected to action
        :type guids1: list
    
        :param action: Action to perform (either START, STOP or DEPLOY)
        :type action: ResourceAction
    
        :param guids2: List of guids of RMs to we waited for
        :type guids2: list
    
        :param state: State to wait for on RMs of list guids2 (STARTED,
            STOPPED, etc)
        :type state: ResourceState
    
        :param time: Time to wait after guids2 has reached status 
        :type time: string

\end{lstlisting}

To use the \textit{register\_condition} method we will need to import the 
ResourceState and the ResourceAction classes

\begin{lstlisting}[language=Python]

from nepi.execution.resource import ResourceState, ResourceAction

\end{lstlisting}

Then, we register the two applications. The first application will 
wait for 5 seconds and the create a file in the host called "greetings" 
with the content "HELLO WORLD". 
The second application will read the content of the file and output it 
to standard output. If the file doesn't exist il will instead output the 
string "FAILED".

\begin{lstlisting}[language=Python]

app1 = add_app(ec, "sleep 5; echo 'HELLO WORLD!' > ~/greetings", node)
app2 = add_app(ec, "cat ~/greetings || echo 'FAILED'", node)

\end{lstlisting}

In order to guarantee that the second application is successful, we need to 
make sure that the first application is executed first. For this we register 
a condition:

\begin{lstlisting}[language=Python]

ec.register_condition (app2, ResourceAction.START, app1, ResourceState.STOPPED)

\end{lstlisting}

We then deploy the two application:

\begin{lstlisting}[language=Python]

ec.deploy(guids=[app1,app2])

\end{lstlisting}

Finally, we retrieve the standard output of the second application, 
which should return the string "HELLO WORLD!".

\begin{lstlisting}[language=Python]

ec.trace(app2, "stdout")

\end{lstlisting}

