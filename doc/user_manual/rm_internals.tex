%%%%%%%%%%%%%%%%%%%%%%%%%%%%%%%%%%%%%%%%%%%%%%%%%%%%%%%%%%%%%%%%%%%%%%%%%%%%%%%
%
%    NEPI, a framework to manage network experiments
%    Copyright (C) 2013 INRIA
%
%    This program is free software: you can redistribute it and/or modify
%    it under the terms of the GNU General Public License as published by
%    the Free Software Foundation, either version 3 of the License, or
%    (at your option) any later version.
%
%    This program is distributed in the hope that it will be useful,
%    but WITHOUT ANY WARRANTY; without even the implied warranty of
%    MERCHANTABILITY or FITNESS FOR A PARTICULAR PURPOSE.  See the
%    GNU General Public License for more details.
%
%    You should have received a copy of the GNU General Public License
%    along with this program.  If not, see <http://www.gnu.org/licenses/>.
%
% Author: Alina Quereilhac <alina.quereilhac@inria.fr>
%
%%%%%%%%%%%%%%%%%%%%%%%%%%%%%%%%%%%%%%%%%%%%%%%%%%%%%%%%%%%%%%%%%%%%%%%%%%%%%%%

%ResourceManger internals

\begin{itemize}
  \item States
  \item Actions
  \item RM API (the schedule methods and the internal methods do\_blah, the states and times are recorded)
  \item failtrap
  \item RM inheritance
  \item init\_copy
  \item Adding new attributes
  \item Traces and the trace method
  \item The start\_with\_condition method and the conditions structure 
  \item How to enforce internal conditions by re-scheduling
\end{itemize}

