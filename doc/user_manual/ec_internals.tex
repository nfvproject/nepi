%%%%%%%%%%%%%%%%%%%%%%%%%%%%%%%%%%%%%%%%%%%%%%%%%%%%%%%%%%%%%%%%%%%%%%%%%%%%%%%
%
%    NEPI, a framework to manage network experiments
%    Copyright (C) 2013 INRIA
%
%    This program is free software: you can redistribute it and/or modify
%    it under the terms of the GNU General Public License as published by
%    the Free Software Foundation, either version 3 of the License, or
%    (at your option) any later version.
%
%    This program is distributed in the hope that it will be useful,
%    but WITHOUT ANY WARRANTY; without even the implied warranty of
%    MERCHANTABILITY or FITNESS FOR A PARTICULAR PURPOSE.  See the
%    GNU General Public License for more details.
%
%    You should have received a copy of the GNU General Public License
%    along with this program.  If not, see <http://www.gnu.org/licenses/>.
%
% Author: Alina Quereilhac <alina.quereilhac@inria.fr>
%
%%%%%%%%%%%%%%%%%%%%%%%%%%%%%%%%%%%%%%%%%%%%%%%%%%%%%%%%%%%%%%%%%%%%%%%%%%%%%%%

% ExperimentController internals

\begin{itemize}
    \item the RMs dictionary
    \item The scheduling API
    \item The scheduler queue, the tasks dictionary, the schedule method 
    \item the processing thread and the \_process method, the inmediate execution queueu and the ParallelRunner
    \item the \_execute method 
    \item The deploy method (implementation), deployment groups
    \item The FailManager and what happens upon release (critical attribute)
\end{itemize}
